\section{Fazit} 
In diesem Versuch ist mithilfe einer Fast-Slow-Koinzidenzschaltung die durchschnittliche Lebensdauer $\tau$ des angeregten $\frac{5}{2}^+$-Zustands von $^{133}_{55}$Cs bestimmt worden. Die Messschaltung verwendet zwei Szintillatoren und Photomultiplier-Tuben, um die Zerfallsprodukte zu detektieren und zu ihrer Energie proportionale Spannungspulse auszugeben. Am Slow-Schaltkreis wird mithilfe zweier SCAs (Single Channel Analyzer) geprüft, ob zwei zeitlich koinzidente Pulse der bevölkernden ($\SI{356}{keV}$) und abregenden ($\SI{81}{keV}$) Zustandsänderung des Isotops zugeordnet werden können. Gleichzeitig erstellt der Fast-Schaltkreis mit einem TAC (Time Amplitude Converter) eine zur Verzögerungszeit zwischen den zwei Signalen (Start-/Stoppsignal) proportionale Spannung. Wird diese mit einem MCA (Multi Channel Analyzer) als Histogramm aufgetragen, kann die Anpassung einer auf Basis der Zeitauflösung des Aufbaus modifizierten Exponentialfunktion erfolgen. Diese angepasste Kurve enthält die durchschnittliche Lebensdauer $\tau$ als Fitparameter.\\

Bevor es jedoch zu dieser Lebensdauermessung kommen kann, erfolgen neben der Einstallunge nötiger Koinzidenzen eine Energie- sowie Zeitkalibration des verwendeten MCA. Dieser arbeitet mit $2^13 = 8192$ diskreten Energiekanälen, welche als Bins für ein Amplituden-Histogramm dienen. Zunächst ist jedoch nicht bekannt, wie die Zuordnung dieser Kanäle im Zusammenhang mit der Energie oder Verzögerungszeit steht. Die Energiekalibration wird durchgeführt, indem bestimmte Maxima im Spektrum von $^{133}$Ba und $^{22}$Na ihren bekannten Übergangsenergien zugeordnet werden und an die resultierenden Energie-Kanal-Zuordnungen eine lineare Funktion angepasst wird (vgl. Abb. \ref{fig:Plot_Energie_Kalibration}). Für die Zeitkalibration wird das Ausgangssignal des TAC als Histogramm aufgetragen, während die Verzögerung zwischen Start- und Stoppsignal in diskreten $\SI{16}{ns}$-Schritten erhöht wird. Dies gibt die sog. Promptkurve, festgehalten in Abb. \ref{fig:Plot_Promptkurve}. Auch hier wird eine lineare Anpassung durchgeführt (vgl. Abb. \ref{fig:Plot_Zeitkalibration}).\\

Nachdem alle diese nötigen Kalibrations- und Koinzidenzmessungen geschehen sind, kann die Messung der Lebensdauerkurve gestartet werden. Das dabei entstehende Histogramm ist in Abb. \ref{fig:Plot_Lebenskurve} mit der dazugehörigen Fitfunktion gezeigt. Die daraus ermittelte Lebensdauer $\tau$ beträgt $\SI{8.48 +- 0.04}{ns}$; als Halbwertszeit formuliert sind es $t_{1/2} = \SI{5,88(02)}{ns}$, was im Hinblick auf die niedrige Anpassungs-Unsicherheit in Kontrast zum in \cite{Praktikumsanleitung} gegebenen Literaturwert von $t_{1/2,\text{lit}} = \SI{6.283(14)}{ns}$ steht. Diese Diskrepanz ist vermutlich teilweise dadurch bedingt, dass die Auswertung die Effekte zufälliger Koinzidenzen, welche die gemessene Lebensdauer nur verkürzen können, nicht berücksichtigt. \\

Damit wurde Versuch 525 \enquote{Nukleare Elektronik} erfolgreich abgeschlossen.