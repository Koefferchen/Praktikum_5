\appendix
\section{Anhang}

    
    \begin{figure}[H]
        \centering
        \includegraphics[width=0.6\textwidth]{figs/Skizze_Driftkammer_Proto.png}
        \caption{Skizze der Auslese-Elektronik der verwendeten Prototyp-Driftkammer \cite{Hammann}. Die Driftkammer besitzt etwa $50$ Drähte, welche mit $3$ Frontends bei einer Zeitauflösung von \SI{2.5}{ns} ausgelesen werden. Die Auslese-Elektronik verstärkt, diskriminiert, versetzt die Signale zeitlich und leitet sie an den Concentrator weiter, welcher die Signale sammelt und über ein Glasfaser-Kabel zur Auswertung an den Computer überträgt. Glasfaserkabel sind weniger störungsanfällig und besitzen ein höhere Bandbreite als draht-basierte Kabel. }
        \label{fig:Skizze_Driftkammer_Proto}
    \end{figure}
    
    

    \begin{figure}[H]
        \centering
        \includegraphics[width=0.8\textwidth]{figs/langzeitmessung_drahtkorrelation.png}
        \caption{2D-Histogramm zur Visualisierung der Korrelation einzelner Draht-Ansprecher (ohne Filterung $\textit{ToT} \geq \SI{100}{ns}$)..}
        \label{fig:langzeitmessung_drahtkorrelation}
    \end{figure}
    
    
    
    \begin{figure}[H]
        \centering
        \includegraphics[width=0.8\textwidth]{figs/langzeitmessung_orts_driftzeit_beziehung.png}
        \caption{Orts-Driftzeit-Beziehung, errechnet als laufende Summe der \textit{NICHT} ToT-gefilterten Driftzeitverteilung (Abb. \ref{fig:langzeitmessung_driftzeit}) mit anschließender Normierung auf \SI{8.5}{mm}.}
        \label{fig:langzeitmessung_orts_driftzeit_beziehung}
    \end{figure}
    
    
    
    \begin{figure}[H]
        \centering
        \includegraphics[width=0.8\textwidth]{figs/langzeitmessung_abstandssumme_vs_differenz_rechts.png}
        \caption{Histogramm der Abstandssumme gegen die Abstandsdifferenz für Drähte mit $32 \leq n_1 \leq 49$. Es fällt auf, dass sich insbesondere bei hoher Abstandssumme die Ereignisse häufen.}
        \label{fig:langzeitmessung_abstand_summe_vs_differenz_rechts}
    \end{figure}