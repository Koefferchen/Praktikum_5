\section{Aufbau und Betrieb des Experiments} \label{sec:Aufbau}
    Für das vorliegende Experiment mit der Prototyp-Driftkammer des B1-Spektrometers \cite{Hammann} liegt die Drahtanordnung wie in Abbildung \ref{fig:Skizze_Drahtanordnung} dargestellt vor, wobei die Drähte senkrecht zur Bildebene sind. Die Kammer ist oben und unten mit zwei Holzplatten verkleidet, worauf der Szintillations-Detektor platziert wird. Der sensitive Bereich des Szintillations-Detektors besteht aus einem länglichen Szintillator-Blatt mit zwei integrierten Lichtleitern, welche Lichtwellen durch Totalreflektion zum Photomultiplier-Rohr leiten.
    
    \begin{figure}[H]
        \centering
        \includegraphics[width=0.9\textwidth]{figs/Skizze_Drahtanordnung.png}
        \caption{Skizze der Drahtanordnung für die verwendete Prototyp-Driftkammer des B1-Spektrometers \cite{Drahtanordnung}. Jede Driftzelle besteht aus einem Anoden-Draht (rot), welcher von einem Hexagon aus Kathoden-Drähten (blau) umgeben wird. Die Driftkammer besteht aus zwei Reihen von Driftzellen, welche zur Minimierung von Randeffekten von geerdeten Drähten (gelb) umgeben sind. Die Anoden- und Kathoden-Drähte haben Durchmesser von \SI{30}{\micro \meter} und \SI{100}{\micro \meter}, wodurch der Lawineneffekt um die Anoden-Drähte optimiert wird.}
        \label{fig:Skizze_Drahtanordnung}
    \end{figure}
    
    Bei der Gasfüllung der vorliegenden Driftkammer handelt es sich um ein Gasgemisch, welches zu \SI{82}{\percent} aus dem Edelgas Argon und zu \SI{18}{\percent} aus dem Löschgas CO$_2$ besteht. Argon eignet sich für die Stoß-Ionisation durch hoch-relativistische Teilchen (hier: Myonen), da es als Edelgas keine Moleküle bildet und die nötige Aktivierungsenergie von \SI{26}{eV} zur Elektron-Ion-Paarbildung \cite[S. 131]{Leo} sicher erreicht wird. Das Löschgas Kohlenstoff-Dioxid verhindert die Sättigung der Driftkammer, da es einen Teil der Energie der emittierten Photonen in den Schwingungs- und Rotations-Freiheitsgraden seiner Moleküle absorbiert. Das Gasgemisch wird aus einer Gasflasche über ein druckregelndes Ventil mit Atmosphärendruck in die Driftkammer geleitet und durchspült diese während des gesamten Versuchs. Das austretende Gas wird durch einen \textit{Silikonöl-Bubbler} geleitet, um den Fluss von Luft in die Driftkammer zu unterbinden. Da von dem Gas keine Gefahr ausgeht, wird es nach dem Durchgang durch die Driftkammer nach draußen geleitet. \cite{Hammann} \\
    
    Die Signale der $48$ Anoden-Drähte werden auf insgesamt $3$ \textit{Frontend}-Karten verteilt, wo diese verstärkt, diskriminiert und zeitlich versetzt werden können (vgl. \ref{sec:ausleseelektronik}). Anschließend werden die Signale aller Frontends in einem \textit{Concentrator} zusammengeführt, welcher die Signale gesammelt über ein Glasfaser-Kabel zur Auswertung an den Computer übergibt (vgl. Anhang Abb. \ref{fig:Skizze_Driftkammer_Proto}). \cite{Hammann, Praktikumsanleitung}
    Der Driftkammer-Strom wird ebenso durch einen an die Driftkammer angebauten \SI{1000(1)}{k\Omega}-Widerstand geleitet. Dieser Widerstand erlaubt den Abgriff der über den Widerstand abfallenden Spannung, um einen Rückschluss auf den Gesamtstrom der Driftkammer schließen zu können.\\
    
    Zur Durchführung einer Messung wird das ROOT-Programm \texttt{fpexperiment} ausgeführt, welches so lange misst, bis der übergebene Zielwert an Ereignissen $N$ erreicht ist. Zur Bestimmung der Ereignisrate kann die Dauer einer Messung mit dem Befehl \texttt{time fpexperiment} im Linux-Terminal ausgegeben werden. Die Daten aller Ereignisse werden automatisch in einer \texttt{.root}-Datei abgelegt, welche im weiteren Verlauf analysiert wird. \cite{Praktikumsanleitung}
