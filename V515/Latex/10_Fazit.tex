\section{Fazit}
    In diesem Versuch ist die Funktionsweise einer Gas-Driftkammer untersucht worden, mit dem Ziel, die Winkelverteilung von Myonen in der kosmischen Strahlung zu messen. Zunächst wurden die analogen Ausgangssignale der Driftkammer und des dazugehörigen Szintillationstriggers ausgemessen sowie der Kammerstrom in Abhängigkeit der Spannung an den Driftzellen-Drähten (mit und ohne zusätzliche $^{90}Sr$-Probe als Elektronenquelle) untersucht. \\
    
    Im Anschluss sind die Betriebsparameter des Aufbaus --- die an der Driftkammer und am Szintillationstrigger angelegte Hochspannung sowie die Zeitverschiebung und Diskriminatorschwelle der Ausleseelektronik --- für eine optimale Messung kalibriert worden. Die optimalen Einstellungen zur Detektion kosmischer Myonen wurden zu $U\sub{sz}^\mu = \SI{1550}{V}$, $U\sub{dk} = \SI{2850}{V}$, \texttt{0x1F DDC} und \texttt{0x18 THR} bestimmt (vgl. Abschnitt \ref{sec:aufbau_kalibration}). Mit diesen Einstellungen ist eine Langzeitmessung der kosmischen Hintergrundstrahlung gestartet worden, welche \SI{100000}{} Ereignisse umfasste. \\

    Die Daten dieser Langzeitmessung sind über verschiedene Auswertungsschritte in die Winkelverteilung der kosmischen Hintergrundstrahlung umgerechnet worden. Zunächst ist das Driftzeitspektrum betrachtet worden, welches durch ein unerwartet hohes Vorkommen langer Driftzeiten auffiel (Abb.\ref{fig:langzeitmessung_driftzeit}) und mutmaßlich durch das geringe \textit{Signal to Noise Ratio} der Myonendetekion erklärt werden kann. Im nächsten Schritt, der Gegenüberstellung von Driftzeit und \textit{Time over Threshold} (ToT), sind die unerwünschten Ereignisse durch Anwendung des Filters $\text{ToT} \geq \SI{100}{ns}$ fast vollständig entfernt worden. Diese Filterung wird für alle kommenden Teilaufgaben beibehalten. \\
    
    Im nächsten Schritt ist die Nummerierung der Drähte der Driftkammer untersucht worden. Anhand der Korrelationen zwischen Ansprechern verschiedener Drähte (Abb. \ref{fig:langzeitmessung_drahtkorrelation_filtered}) konnte die Nummerierung der Drähte bis auf einen Spiegelungs-Freiheitsgrad bestimmt werden (vgl. Abb. \ref{fig:zellen_nummerierung_default}). Die Nummerierung wurde nun für den weiteren Versuch so abgeändert (Abb. \ref{fig:zellen_nummerierung_neu}), dass benachbarte Drahtnummern maximale Korrelation aufweisen (vgl. Abb. \ref{fig:langzeitmessung_drahtkorrelation_filtered_renumbered}). \\
    
    Mit einigen vereinfachenden Annahmen über die Geometrie der Driftkammer (insbesondere der, dass alle Anodendrähte in gleichmäßigem Abstand in einer Ebene liegen), konnte durch laufende Summierung des Driftzeitspektrums eine Orts-Driftzeit-Beziehung bestimmt werden, welche einer gemessenen Driftzeit einen bestimmten Minimalabstand des Myons zum Draht zuweist (vgl. Abschnitt \ref{sec:orts_driftzeit_beziehung}). Mit dieser Beziehung konnte für die Fälle, in denen benachbarte Zellen dasselbe Myon detektieren, die Summe und Differenz der jeweiligen Drahtabstände bestimmt werden.  Diese sind in 2D-Histogrammen gegeneinander aufgetragen worden. Damit konnten für verschiedene Winkelbereiche in Korrelation mit bestimmten Drahtnummern winkelabhängige Trends ausgelesen werden (vgl. Abschnitt \ref{sec:summe_vs_diff}), die der theoretischen Erwartung entsprechen. \\
    
    Zuletzt konnte die Winkelverteilung der kosmischen Strahlung berechnet und daran eine Funktion angepasst werden. Dafür wurde jedem Draht zunächst durch seine relative Position zum Szintillationstrigger ein Winkel zugeschrieben. In einem Histogramm werden jedem so ermittelten Winkel die gezählten Ansprecher des dazugehörigen Drahtes zugewiesen. Mit der gefundenen Anpassungsfunktion, gegeben in Gl. \ref{eq:winkelverteilung_fit_fazit}, ist der Verlauf aus Abb. \ref{fig:langzeitmessung_winkelverteilung_fit} erstellt worden.
    
    \begin{align} \label{eq:winkelverteilung_fit_fazit}
        N(\Phi) = 1750(10) \cdot \cos^{2,38(3)} \cdot \left(\Phi + 0,007(3)\right)
    \end{align}
    
    Dieser Verlauf zeigt eine gute Anpassung an die gefundenen Daten, insbesondere in Anbetracht der angewandten Vereinfachungen und Approximationen. Somit wurde die Winkelverteilung kosmischer Myonen erfolgreich quantifiziert.