

    \begin{table}[H]
        \centering
        \begin{tabular}{c c c c c c c c} \toprule
            $U_{\mathrm{sz}}$ / V &
            $N_{\mathrm{mit}}$ / 1 &
            $t_{\mathrm{mit}}$ / s &
            $N_{\mathrm{ohne}}$ / 1 &
            $t_{\mathrm{ohne}}$ / s &
            $n_{\mathrm{mit}}$ / 1/s &
            $n_{\mathrm{ohne}}$ / 1/s &
            $\frac{n_{\mathrm{mit}}}{n_{\mathrm{ohne}}}$ / 1 \\ \midrule
    -1500 & 200 & 203,2 & 100 & 153,7 & 0,9843 & 0,6506 & 1,5129 \\
    -1540 & 200 & 82,7  & 100 & 67,1  & 2,4184 & \cellcolor{green!30} 1,4903 & 1,6228 \\
    -1560 &  \text{---}   &  \text{---}     & 200 & 71,6  &  \text{---}      & 2,7933 &  \text{---}      \\
    -1590 & 1000 & 106,2 & 300 & 78,3 & 9,4162 & 3,8314 & 2,4576 \\
    -1630 & 2000 & 91,4 & 400 & 70,1 & 21,8818 & 5,7061 & 3,8348 \\
    -1700 & 5000 & 79,3 & 1000 & 106,1 & 63,0517 & 9,4251 & 6,6898 \\
    -1750 & 10000 & 84,0 & 3000 & 201,1 & 119,0476 & 14,9179 & 7,9802 \\
    -1800 & 20000 & 104,5 & 5000 & 219,7 & 191,3876 & 22,7583 & \cellcolor{blue!30} 8,4096 \\
    -1850 & 40000 & 147,6 & 5000 & 134,3 & 271,0027 & 37,2301 & 7,2791 \\
    -1900 & 100000 & 287,1 & 10000 & 186,0 & 348,3107 & 53,7634 & 6,4786 \\
    -1920 & 100000 & 271,0 & 10000 & 165,6 & 369,0037 & 60,3865 & 6,1107 \\
    -2000 & 100000 & 225,0 & 10000 & 122,1 & 444,4444 & 81,9000 & 5,4267 \\ \bottomrule
        \end{tabular}
        \caption{Meisreihe zur Bestimmung der optimalen Szintillator-Spannungen $U\sub{sz}^e$ zur Messung von Elektronen und $U\sub{sz}^\mu$ zur Messung von Myonen. Es zeigt sich, dass $n\sub{ohne} \approx n\sub{exp}$ in der Region von $U\sub{sz}^\mu = \SI{-1550(10)}{V}$ erreicht wird (grün). Das beste SNR von $\num{8.4096}$ zur Messung von Elektronen hingegen wird bei der Hochspannung $U\sub{sz}^e = \SI{-1800(50)}{V}$ erreicht (blau). Die Unsicherheiten betragen $\Delta U\sub{sz} = \SI{10}{V}$, $\Delta N = \sqrt{N}$ und $\Delta t = \SI{0.1}{s}$.}
        \label{tab:Szintillator_Spannungen}
    \end{table}
