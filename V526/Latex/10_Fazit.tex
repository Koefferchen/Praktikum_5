\section{Fazit} 
In diesem Versuch ist die Compton-Streuung untersucht worden. Dafür wurde eine Cäsiumquelle eingesetzt, die auf ein Target aus Alumninium strahlte, sowie eine Europiumquelle zur Kalibration des Aufbaus. Die davon gestreute bzw. gedämpfte Strahlung wurde mit einem Szintillationsspektrometer, bestehend aus einem NaJ(Tl)-Szintillator und einer Photomultiplier-Röhre, analysiert. Durch Verwendung eines Verstärkers und eines Vielkanalanalysators (MCA) konnten Energie-Histogramme der resultierenden Strahlung erstellt und später ausgewertet werden.\\

Zunächst ist die Abschwächung der Strahlung für verschiedene Target-Dicken zwischen \SI{1}{mm} und \SI{30}{mm} untersucht worden. Mit Ausnahme eines Ausreißers, der vermutlich eine Konsequenz eines unzureichend breiten Targets war, konnte die exponentielle Reduktion der transmittierten Intensität mit der Targetdicke nach Lambert-Beer bestätigt werden. In weitgehender Übereinstimmung wurde der totale Wirkungsquerschnitt der Streuuung von $\SI{661.7}{keV}$ Photonen an Aluminium in barn / Elektron bestimmt:
\begin{align}
    \sigma \sub{tot} = \SI{255.6 +- 3.770}{\barn}
\end{align}
Daraufhin wurde eine Energiekalibration des Aufbaus durchgeführt. Mit abgeschirmter Cäsiumquelle wurde statt einem Aluminiumtarget eine Europiumquelle in den Versuchsaufbau eingesetzt. Das resultierende Spektrum erlaubte die Zuordnung einiger der beobachteten Maxima am MCA zu bekannten Energien verschiedener Europium-Zerfallskanäle. Eine lineare Anpassung an die so gefundenen Datenpunkte diente für die weitere Auswertung als Energiekalibration und verifizierte die korrekte lineare Funktionsweise des MCA. Die Beobachtung der relativen Intensitäten der Europium-Linien, gefolgt von einem Abgleich mit bekannten Übergangswahrscheinlichkeiten, erlaubt die Bestimmung der relativen Detektionseffizienz des verwendeten Detektors für ein paar Energiewerte. Aufgrund der beobachteten Verteilung der gesammelten Effizienz-Datenpunkte ist eine Gaußkurve mit linearem Untergrund angepasst worden, welche jedoch aufgrund der niedrigen Anzahl Datenpunkte und daher kaum motivierten Form einen schlechten Anhaltspunkt für die Detektionseffizienz im restlichen Energiebereich darstellte.\\

Die letzte Messreihe erfasste das Compton-Streuspektrum der Cäsium-Probe --- bei festgesetzter Targetdicke von \SI{1}{mm} --- bei variierender Winkeleinstellung des Szintillationsspektrometers. Dabei wurde ebenso zu jeder Messung eine Hintegrundmessung ohne Target durchgeführt und für die Auswertung von der Streumessung subtrahiert. Die Analyse der gefundenen Spektren erlaubte eine hervorragende Verifizierung der Compton-Energie-Relation (Gl. \ref{eq:E^prime(E)}), indem die im Spektrum beobachteten Maxima ihren jeweiligen Energien zugeordnet und diese gegen den Streuwinkel aufgetragen wurden. Aus der Anpassung wurde die einfallende Photonenenergie als $E_\gamma = \SI{647.8(1.0)}{keV}$ bestimmt, was knapp unter dem Literaturwert von $\SI{661.7}{keV}$ liegt.\\

In der weiteren Auswertung wurden die beobachtete Intensität der Streustrahlung bei allen Winkeleinstellungen mit der erwarteten Verteilung durch den Klein-Nishina-Wirkungsquerschnitt verglichen. In diesen Prozess sind insbesondere aufwendige Intensitäts-Korrekturen aufgrund der Detektoreffizienz und der winkelabhängigen mittleren Abschwächung aufgrund der Weglänge im Target eingeflossen. Beide Korrekturen stießen leider auf unerwartete Hürden, aufgrund derer die Korrekturen nicht ganz der Erwartung entsprachen. Es resultierten zwei Auftragungen der beobachteten Intensitäten gegen den Klein-Nishina-Wirkungsquerschnitt, einmal mit und einmal ohne Einbezug der Detektoreffizienz (Abb. \ref{fig:intensität_vs_kn_unkorrigiert}, \ref{fig:intensität_vs_kn_korrigiert}). Beide Verläufe sind grob mit dem erwarteten Wirkungsquerschnitt vereinbar, weisen aber beide in Teilen Mängel in der Anpassung auf. \\

Somit konnte der totale Wirkungsquerschnitt von Compton-Streuung mit der Quelle $^{137}$Cs quantitativ sowie der Winkelabhängigkeit von Energie und Intensität der gestreuten Strahlung qualitativ nachgewiesen werden. Damit wurde der Versuch \enquote{P526 Compton-Effekt} erfolgreich abgeschlossen.