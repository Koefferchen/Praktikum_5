\section{Anhang}

\begin{table}[H]
    \centering
    \begin{tabular}{SSSS} \toprule
        {Target-Dicke $d$ / \unit{mm}} & {Mittelwert $\mu$ / Kanal} & {Breite $\sigma$ / Kanal} & {Fläche $A$ / 1} \\ \midrule
        \num{30.00(2)} & $\num{4829.1932 +- 0.5060}$ & $\num{141.9069 +- 0.4106}$ & $\num{108344.8387 +- 358.4752}$   \\
        \num{20.16(2)} & $\num{4825.7330 +- 0.5264}$ & $\num{141.2663 +- 0.4301}$ & $\num{100103.9038 +- 345.7141}$   \\
        \num{10.12(2)} & $\num{4824.0328 +- 0.4731}$ & $\num{141.4586 +- 0.3843}$ & $\num{122385.4936 +- 379.5939}$   \\
        $\sqrt{2}\cdot$\num{5.00(2)} & $\num{4830.8099 +- 0.4548}$ & $\num{141.1597 +- 0.3667}$ & $\num{129793.7327 +- 389.5311}$   \\
        \num{5.00(2)} & $\num{4824.3885 +- 0.4458}$ & $\num{140.9955 +- 0.3609}$ & $\num{135390.6505 +- 397.7100}$   \\
        \num{1.00(2)} & $\num{4823.6612 +- 0.4269}$ & $\num{140.9666 +- 0.3433}$ & $\num{147046.2297 +- 413.3962}$   \\
        \num{0} & $\num{4819.9770 +- 0.4238}$ & $\num{140.9118 +- 0.3435}$ & $\num{149311.8281 +- 416.6527}$   \\ \bottomrule
    \end{tabular}
    \caption{Fit-Parameter der Anpassung von Gaußfunktionen nach Gleichung \eqref{eq:Gauss}  für die Photopeaks von Caesium mit jeweiliger effektiver Targeticke $d$ (vgl. Abb. \ref{fig:Ceasium_ALLmm}).}
    \label{tab:TotWirk_Gauss_Params}
\end{table}



\begin{table}[H]
    \centering
    \begin{tabular}{SSSS} \toprule
{Erwartete Energie $E$ / \unit{keV}} & {Mittelwert $\mu$ / Kanal} & {Breite $\sigma$ / Kanal} & {Fläche $A$ / 1} \\
\midrule
{---} & $\num{335.0545 +- 0.2462}$ & $\num{31.3539 +- 0.2187}$ & $\num{25530.9324 +- 183.5854}$   \\
{---} & $\num{616.9561 +- 3.6235}$ & $\num{66.0695 +- 4.4893}$ &  $\num{3718.6370 +- 191.9845}$   \\
121.7817 & $\num{949.3850 +- 0.6665}$ & $\num{47.9362 +- 0.6781}$ & $\num{14731.6250 +- 180.1867}$   \\
244.6974 & $\num{1822.0737 +- 4.6342}$ & $\num{134.8087 +- 4.9693}$ & $\num{7382.2073 +- 224.5096}$   \\
344.2785 & $\num{2537.7899 +- 2.0822}$ & $\num{116.2500 +- 2.1182}$ & $\num{13112.8091 +- 205.2308}$   \\
778.9045 & $\num{5564.0882 +- 7.7562}$ & $\num{239.3552 +- 10.2584}$ & $\num{5012.5643 +- 158.5282}$   \\
964.057 & $\num{7732.1996 +- 5.9477}$ & $\num{187.6605 +- 5.8494}$ & $\num{3747.2745 +- 102.8282}$  \\ \bottomrule
    \end{tabular}
 \caption{Fit-Parameter der Anpassung von Gaußfunktionen nach Gleichung \eqref{eq:Gauss}  für die Photopeaks von Europium, zugeordnet zum Literaturwert der Energie (vgl. Tab. \ref{tab:Eu_Zerfallslinien}).}
    \label{tab:Eu_Fitparameter_Kalib}
\end{table}





\begin{table}[H]
    \centering
    \begin{tabular}{c|c} \toprule
        $\mu $ & $2041 \pm 22648324$   \\
        $\sigma $ & $183 \pm 14911827$   \\
        $A $ & $465 \pm 85309764$   \\
        $c $ & $\num{27274.8398 +- 7275.6147}$   \\
        $d $ & $\num{15603.2168 +- 2290.7750}$ \\ \bottomrule
    \end{tabular}
    \caption{Anpassungs-Parameter der empirischen Fitfunktion $\epsilon \sub{Fit}$ \eqref{eq:epsilon_eff_kurve}. Aufgrund des Mangels an Datenpunkten liefert die Fitfunktion hauptsächlich Informationen über den qualitativen Verlauf der Effizienz als Funktion des Kanal-Indexes $K$. Die hohen Unsicherheiten der Parameter der Gaußfunktion $G(K,\mu,\sigma,A)$ weisen auf die fehlende Gewissheit bezüglich Position, Höhe und Verlauf des Maximums hin.}
    \label{tab:Effizienz_Fitparameter}
\end{table}



\begin{figure}[H]
    \centering
    \includegraphics[height = 0.35\textheight]{figs/angle_spectrum_fits/Spektrum_035_grad_Fit.png}
    \caption{Differenz zwischen Streuspektrum und Hintergrundmessung in einem Winkel von \SI{35}{\degree}. Neben dem eindeutig erkennbaren Peak, der der gesuchten Compton-Energie entspricht, ist ein intensives Rauschen zu erkennen. Die Amplitude des Rauschens fällt rechts vom Compton-Maximum nicht so plötzlich ab wie bei höheren Winkeln.}
    \label{fig:streuspektrum_fit_beispiel_35}
\end{figure}
\begin{figure}[H]
    \centering
    \includegraphics[height = 0.35\textheight]{figs/angle_spectrum_fits/Spektrum_120_grad_Fit.png}
    \caption{Differenz zwischen Streuspektrum und Hintergrundmessung in einem Winkel von \SI{120}{\degree}. Neben dem eindeutig erkennbaren Peak, der der gesuchten Compton-Energie entspricht, ist ein intensives Rauschen zu erkennen. Dieses weist eine starke und plötzliche Reduktion im höherenergetischen Bereich auf, welche sich gut mit der Position des \SI{662}{keV}-Caesium-Peak deckt.}
    \label{fig:streuspektrum_fit_beispiel_120}
\end{figure}

\begin{table}[H]
    \centering
    \begin{tabular}{cccccccccccc}
        \toprule
        $\theta / \si{\degree}$ & $K\sub{min}$ / $1$ & $K \sub{max}$ / $1$ & $A$ / $10$ & $\mu$ / $1$ & $\sigma$ / $1$ & $m$ / $10^{-3}$ & $b$ / $1$ & $E(\mu)$ / \si{keV}\\
        \midrule
        35(1) & 3300 & 4400 & 1226(7) & 3855(6) & 222(9) & -2.8(7) & 11(3) & 527.4(0.9) \\
        40(1) & 3150 & 4150 & 1133(7) & 3659(7) & 223(11) & -2.5(8) & 9(3) & 499.9(0.9)\\
        45(1) & 2900 & 4000 & 949(4) & 3463(5) & 203(7) & -1.5(5) & 5.9(1.8) & 472.6(0.7)\\
        50(1) & 2800 & 3700 & 918(8) & 3276(7) & 205(11) & -1.9(9) & 6(3) & 446.4(1.0)\\
        55(1) & 2600 & 3500 & 825(7) & 3099(7) & 194(11) & -4(1) & 11(3) & 421.7(1.0)\\
        60(1) & 2400 & 3400 & 755(4) & 2910(6) & 185(8) & -1.9(6) & 6(2) & 395.3(0.8)\\
        65(1) & 2300 & 3200 & 650(5) & 2763(6) & 170(9) & -1.6(8) & 5(2) & 374.7(0.9)\\
        70(1) & 2150 & 3000 & 669(5) & 2614(6) & 158(8) & -1.6(1.0) & 4(3) & 354.0(0.9)\\
        75(1) & 2050 & 2900 & 631(5) & 2472(7) & 161(9) & -1.8(1.0) & 5(3) & 334.2(1.0)\\
        80(1) & 1950 & 2700 & 612(6) & 2363(7) & 145(10) & -4.5(1.5) & 11(3) & 318.9(0.9)\\
        85(1) & 1850 & 2600 & 456(4) & 2238(7) & 127(9) & -0.3(1.3) & 2(3) & 301.4(0.9)\\
        90(1) & 1750 & 2500 & 442(4) & 2116(7) & 125(9) & 0.2(1.3) & 1(3) & 284.4(1.0)\\
        95(1) & 1650 & 2400 & 418(3) & 2019(6) & 108(7) & -1.0(1.3) & 4(3) & 270.8(0.8)\\
        100(1) & 1550 & 2300 & 591(4) & 1951(6) & 121(7) & -3.6(1.4) & 7(3) & 261.3(0.8)\\
        105(1) & 1500 & 2200 & 555(4) & 1867(6) & 113(6) & -2.0(1.6) & 4(3) & 249.6(0.8)\\
        110(1) & 1450 & 2100 & 520(4) & 1791(6) & 94(7) & 0.8(1.8) & 5(4) & 238.9(0.6)\\
        115(1) & 1350 & 2100 & 565(4) & 1732(6) & 98(5) & 1.6(1.4) & -2(4) & 230.8(0.6)\\
        120(1) & 1350 & 2000 & 631(4) & 1672(6) & 99(5) & 2.0(2.0) & -4(4) & 222.3(0.6) \\
        \bottomrule
    \end{tabular}
    \caption{Anpassungsparameter für die Gaußkurven mit linearem Untergrund (vgl. Gl. \ref{eq:Gauß_Linear}) an die winkelabhängigen Streuspektren. Hierbei beschreibt $A$ die Intensität, $\mu$ den Peak-Schwerpunkt, $\sigma$ die Breite des Pekas und $(m,b)$ die Steigung und den Versatz des linearen Hintergrunds. Die Anpassung werden jeweils auf den Intervallen $K \in [K\sub{min},K\sub{max}]$ um $\mu$ durchgeführt. Die letzte Spalte zeigt die Umrechnung des Peak-Schwerpunktes gemäß Energie-Kalibration  \eqref{eq:energiekalibration}. Zur Erhöhung der Lesbarkeit wird $A$ in Einheiten von $10^1$ Kanälen und $m$ von $10^{-3}$ Kanälen angegeben.}
    \label{tab:streuspektren_fitparameter_gauß_linear}
\end{table}

\begin{table}[H]
        \centering
        \begin{tabular}{ccc}
            \toprule
            $\theta$ / \si{\degree} & $I_{rel}$, unkorrigiert / $1$ & $I_{rel}$, korrigiert / $1$ \\
            \midrule
            35(1) & 1.00(6) & 1.00(6) \\
            40(1) & 0.92(7) & 0.91(7) \\
            45(1) & 0.77(3) & 0.75(3) \\
            50(1) & 0.75(7) & 0.71(6) \\
            55(1) & 0.67(6) & 0.63(6) \\
            60(1) & 0.62(4) & 0.57(3) \\
            65(1) & 0.53(4) & 0.48(4) \\
            70(1) & 0.55(4) & 0.49(4) \\
            75(1) & 0.51(4) & 0.41(3) \\
            80(1) & 0.50(5) & 0.31(3) \\
            85(1) & 0.37(3) & 0.15(1) \\
            90(1) & 0.36(3) & 0.11(1) \\
            95(1) & 0.34(3) & 0.10(1) \\
            100(1) & 0.48(4) & 0.15(1) \\
            105(1) & 0.45(4) & 0.17(1) \\
            110(1) & 0.42(3) & 0.20(1) \\
            115(1) & 0.46(3) & 0.26(2) \\
            120(1) & 0.52(3) & 0.35(2) \\
            \bottomrule
        \end{tabular}
        \caption{Extrahierte relative Intensitäten der Compton-Streuung, mit und ohne Effizienzkorrektur nach Abschnitt \ref{sect:energieeffizienz}. Die Weglängenkorrektur ist eingerechnet, die Normierung auf $I_{rel}(\theta = \SI{35}{\degree})$ ist in beiden Fällen separat geschehen.}
        \label{tab:compton_intensitäten}
\end{table}

Analytische Stammfunktion des Integrals in Gl. \ref{eq:target_abschwächung}:

\begin{align} \label{eq:stammfunktion_weglänge}
    \begin{split}
        \mathcal{I}(\theta) = &\frac{\left(\frac{2 \left|\sin\left({\theta}\right)\right| \left(\frac{d - 1}{\left|\cos\left({\theta}\right)\right|} - \frac{b}{2 \left|\sin\left({\theta}\right)\right|}\right)}{\left|\frac{d - 1}{\left|\cos\left({\theta}\right)\right|} - \frac{b}{2 \left|\sin\left({\theta}\right)\right|}\right|} - 4 \left|\cos\left({\theta}\right)\right| \left|\sin\left({\theta}\right)\right| - 2 \left|\sin\left({\theta}\right)\right|\right) \mathrm{e}^{-{\mu}d}}{4{\mu} \left(1 - \mathrm{e}^{-{\mu}}\right) \left|\cos\left({\theta}\right)\right| \left|\sin\left({\theta}\right)\right|} \\
        &- \frac{\left(4 \left|\cos\left({\theta}\right)\right| \left|\sin\left({\theta}\right)\right| d + 2 \left|\sin\left({\theta}\right)\right| \left(d - 1\right) - 2 \left|\cos\left({\theta}\right)\right| \left|\sin\left({\theta}\right)\right| \left|\frac{d - 1}{\left|\cos\left({\theta}\right)\right|} - \frac{b}{2 \left|\sin\left({\theta}\right)\right|}\right| + b \left|\cos\left({\theta}\right)\right|\right) \mathrm{e}^{-{\mu}d}}{4 \left(1 - \mathrm{e}^{-{\mu}}\right) \left|\cos\left({\theta}\right)\right| \left|\sin\left({\theta}\right)\right|}
    \end{split}
\end{align}
