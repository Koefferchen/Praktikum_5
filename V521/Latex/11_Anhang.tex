\section{Anhang}


    % \textbf{Abbildungen nebeneinander:}
    % \begin{figure}[H]
    %     \centering
    %     \begin{subfigure}{0.4\textwidth}
    %         \includegraphics[width=\linewidth]{figs/V4_2e_emitter.jpg}
    %         \caption{Linke Abbildung}
    %     \end{subfigure}
    %     \hspace{1cm}
    %     \begin{subfigure}{0.4\textwidth}
    %         \includegraphics[width=\linewidth]{figs/V4_2e_kaskode.jpg}
    %         \caption{Rechte Abbildung}
    %         \label{fig:frequ_kaskode}
    %     \end{subfigure}
    %     \caption{Gesamtabbildung}
    % \end{figure}
    

    % \textbf{Text neben Abbildung:}
    % \begin{figure}[H]
    %     \centering
    %     \begin{minipage}{0.4\textwidth}
    %         \centering
    %         \includegraphics[width=\linewidth]{figs/V4_2e_emitter.jpg}
    %         \caption{Abbildung neben Text}
    %     \end{minipage} 
    %     \hspace{1cm}
    %     \begin{minipage}{0.4\textwidth}
    %         Text neben Abbildung... \cite{Praktikumsanleitung}
    %     \end{minipage}
    % \end{figure}
    
    
    % \textbf{Formatierung von Zahlen, Einheiten und Unsicherheiten:}
    % \begin{align}
    %      E_1 &= \SI{511 +- 0.1}{keV} \\
    %      \text{unit of}(E_1) &= \unit{keV} \\
    %      \text{value of}(E_1) &= \num{511 +- 0.1}
    % \end{align}
    
    % \textbf{Formatierung von Tabellen mit Messwerten:}
    % \begin{table}[H]
    %     \centering
    %     \begin{tabular}{|c|c|c|} \hline
    %         $\mu$ [\unit{keV}] & $\sigma$ [\unit{keV}] & I [1] \\ \hline
    %         \num{1.0 +- 0.2} & \num{1.0 +- 0.2} & \num{1.0 +- 0.2} \\
    %         \num{1.0 +- 0.2} & \num{1.0 +- 0.2} & \num{1.0 +- 0.2} \\ \hline
    %     \end{tabular}
    %     \caption{}
    %     \label{tab:}
    % \end{table}
    
    
    \begin{table}[H]
        \centering
        \begin{tabular}{SSSSSS} \toprule
         	{$E_{lit}$ / keV} & {$\mu$ / Kanal} & {$\sigma \sub{FWHM}$ / Kanal} & {$A$ / 1} & {$a$ / Kanal$^{-1}$} & {$b$ / 1} \\ \midrule
        	661.7000 +- 0.0000 & 11902.5296 +- 661.7000 & 19.5102 +- 11902.5296 & 18968.7546 +- 19.5102 & -0.1395 +- 18968.7546 & 1674.6426 +- 0.1395 \\
        	1173.2000 +- 0.0000 & 13518.7934 +- 1173.2000 & 20.4113 +- 13518.7934 & 16970.5550 +- 20.4113 & -0.0808 +- 16970.5550 & 1098.2638 +- 0.0808 \\
        	1332.5000 +- 0.0000 & 6711.8614 +- 1332.5000 & 16.3792 +- 6711.8614 & 82892.4068 +- 16.3792 & -0.3629 +- 82892.4068 & 2453.9008 +- 0.3629 \\
        	121.7817 +- 0.0000 & 1234.2248 +- 121.7817 & 12.1629 +- 1234.2248 & 55403.1802 +- 12.1629 & -0.0466 +- 55403.1802 & 205.4848 +- 0.0466 \\
        	244.6974 +- 0.0000 & 2481.3803 +- 244.6974 & 13.3241 +- 2481.3803 & 12758.2927 +- 13.3241 & -0.0954 +- 12758.2927 & 287.2933 +- 0.0954 \\
        	344.2785 +- 0.0000 & 3491.8329 +- 344.2785 & 14.1253 +- 3491.8329 & 32026.3663 +- 14.1253 & -0.2504 +- 32026.3663 & 912.2538 +- 0.2504 \\
        	411.1165 +- 0.0000 & 4169.7665 +- 411.1165 & 14.3700 +- 4169.7665 & 2305.7296 +- 14.3700 & 0.0028 +- 2305.7296 & 10.1491 +- 0.0028 \\
        	443.9606 +- 0.0000 & 4503.3662 +- 443.9606 & 14.8544 +- 4503.3662 & 2914.8297 +- 14.8544 & -0.0332 +- 2914.8297 & 171.0596 +- 0.0332 \\
        	778.9045 +- 0.0000 & 7901.4100 +- 778.9045 & 16.7408 +- 7901.4100 & 6903.0107 +- 16.7408 & -0.0429 +- 6903.0107 & 355.0820 +- 0.0429 \\
        	867.3800 +- 0.0000 & 8799.4136 +- 867.3800 & 18.0662 +- 8799.4136 & 2178.8990 +- 18.0662 & -0.0037 +- 2178.8990 & 46.7165 +- 0.0037 \\
        	964.0570 +- 0.0000 & 9780.1727 +- 964.0570 & 18.4596 +- 9780.1727 & 6471.6482 +- 18.4596 & -0.0285 +- 6471.6482 & 288.1395 +- 0.0285 \\
        	1085.8370 +- 0.0000 & 11015.5290 +- 1085.8370 & 17.7794 +- 11015.5290 & 3813.9359 +- 17.7794 & 0.0965 +- 3813.9359 & -1049.9444 +- 0.0965 \\
        	1112.0760 +- 0.0000 & 11281.8397 +- 1112.0760 & 18.8507 +- 11281.8397 & 5226.3297 +- 18.8507 & -0.0831 +- 5226.3297 & 946.0066 +- 0.0831 \\
        	1408.0130 +- 0.0000 & 14284.5448 +- 1408.0130 & 20.7786 +- 14284.5448 & 6452.9450 +- 20.7786 & -0.0126 +- 6452.9450 & 181.1879 +- 0.0126 \\ \bottomrule
        \end{tabular}
        \caption{Anpassungs-Parameter der Gaußfits nach Gleichung \eqref{eq:Gauss_mit_Bg} für die Quellen $^{60}$Co, $^{137}$Cs und $^{152}$Eu am HPGe-Detektor. Zeilen 1, 2 und 3 behandeln den Caesium-Peak sowie die beiden Cobalt-Peaks, während die restlichen Gaußfunktionen durch Tabelle \ref{tab:Eu_Literatur_Linien} Europium zuzuordnen sind.}
        \label{tab:Gaussfit_Params_HPGe}
    \end{table}
    
    \begin{table}[H]
        \centering
        \begin{tabular}{SSSSSS} \toprule
         	{$E_{lit}$ / keV} & {$\mu$ / Kanal} & {$\sigma \sub{FWHM}$ / Kanal} & {$A$ / 1} & {$a$ / Kanal$^{-1}$} & {$b$ / 1} \\ \midrule
        	661.7000 +- 0.0000 & 9767.4147 +- 661.7000 & 536.8925 +- 9767.4147 & 37778.4925 +- 536.8925 & -0.0017 +- 37778.4925 & 21.0217 +- 0.0017 \\
        	1173.2000 +- 0.0000 & 11071.4248 +- 1173.2000 & 543.1894 +- 11071.4248 & 30914.2620 +- 543.1894 & -0.0017 +- 30914.2620 & 21.0217 +- 0.0017 \\
        	1332.5000 +- 0.0000 & 5617.7802 +- 1332.5000 & 386.8493 +- 5617.7802 & 174119.6846 +- 386.8493 & -0.0063 +- 174119.6846 & 38.4724 +- 0.0063 \\
        	121.7817 +- 0.0000 & 1130.5634 +- 121.7817 & 110.9213 +- 1130.5634 & 57312.9510 +- 110.9213 & -0.1774 +- 57312.9510 & 302.1499 +- 0.1774 \\
        	244.6974 +- 0.0000 & 2165.7956 +- 244.6974 & 197.4148 +- 2165.7956 & 13773.8374 +- 197.4148 & -0.0383 +- 13773.8374 & 126.1783 +- 0.0383 \\
        	344.2785 +- 0.0000 & 3003.6967 +- 344.2785 & 265.5831 +- 3003.6967 & 38653.3220 +- 265.5831 & -0.0107 +- 38653.3220 & 56.7399 +- 0.0107 \\
        	778.9045 +- 0.0000 & 6576.8535 +- 778.9045 & 366.3641 +- 6576.8535 & 5848.4938 +- 366.3641 & -0.0023 +- 5848.4938 & 26.5105 +- 0.0023 \\
        	964.0570 +- 0.0000 & 8093.3053 +- 964.0570 & 480.4954 +- 8093.3053 & 7001.1034 +- 480.4954 & -0.0043 +- 7001.1034 & 39.3105 +- 0.0043 \\
        	1408.0130 +- 0.0000 & 11643.0497 +- 1408.0130 & 534.9874 +- 11643.0497 & 5656.1291 +- 534.9874 & -0.0005 +- 5656.1291 & 6.0784 +- 0.0005 \\ \bottomrule
        \end{tabular}
        \caption{Anpassungs-Parameter der Gaußfits nach Gleichung \eqref{eq:Gauss_mit_Bg} für die Quellen $^{60}$Co, $^{137}$Cs und $^{152}$Eu am NaI-Detektor. Zeilen 1, 2 und 3 behandeln den Caesium-Peak sowie die beiden Cobalt-Peaks, während die restlichen Gaußfunktionen durch Tabelle \ref{tab:Eu_Literatur_Linien} Europium zuzuordnen sind.}
        \label{tab:Gaussfit_Params_NaI}
    \end{table}
    
    
    \begin{figure}[H]
        \centering
        \begin{subfigure}{0.4\textwidth}
            \includegraphics[width=\linewidth]{figs/HPGe_bg.jpg}
        \end{subfigure}
        \hspace{0.5cm}
        \begin{subfigure}{0.4\textwidth}
            \includegraphics[width=\linewidth]{figs/NaI_bg.jpg}
        \end{subfigure}
        \caption{Untergrund-Messungen am HPGe-Detektor (links) sowie am NaI-Detektor (rechts). Wie erwartet fallen die Zählraten über die Messzeit $T = \SI{300}{s}$ in beiden Fällen gering aus.}
        \label{fig:Backgrounds}
    \end{figure}
    
    \begin{figure}[H]
        \centering
        \begin{subfigure}{0.4\textwidth}
            \includegraphics[width=\linewidth]{figs/HPGe_Co_1173.jpg}
        \end{subfigure}
        \hspace{0.5cm}
        \begin{subfigure}{0.4\textwidth}
            \includegraphics[width=\linewidth]{figs/HPGe_Co_1332.jpg}
        \end{subfigure}
        \caption{Vergrößerte Ausschnitte aus dem Cobalt-Spektrum am HPGe-Detektor.}
        \label{fig:spektrum_Co_Zoom}
    \end{figure}
    
    \begin{figure}[H]
        \centering
        \begin{subfigure}{0.4\textwidth}
            \includegraphics[width=\linewidth]{figs/HPGe_Cs_661.jpg}
        \end{subfigure}
        \hspace{0.5cm}
        \begin{subfigure}{0.4\textwidth}
            \includegraphics[width=\linewidth]{figs/HPGe_Eu_5000.jpg}
        \end{subfigure}
        \caption{Vergrößerte Ausschnitte aus dem Caesium- (links) und dem Europium-Spektrum (rechts) am HPGe-Detektor.}
        \label{fig:spektrum_Cs+Eu_Zoom}
    \end{figure}
    
    \begin{figure}[H]
        \centering
        \begin{subfigure}{0.4\textwidth}
            \includegraphics[width=\linewidth]{figs/HPGe_Eu_10000.jpg}
        \end{subfigure}
        \hspace{0.5cm}
        \begin{subfigure}{0.4\textwidth}
            \includegraphics[width=\linewidth]{figs/HPGe_Eu_15000.jpg}
        \end{subfigure}
        \caption{Vergrößerte Ausschnitte aus dem Europium-Spektrum am HPGe-Detektor (2).}
        \label{fig:spektrum_Eu_Zoom_2}
    \end{figure}
    
    

    \newpage
    \begin{figure}[H]
        \centering
        \includegraphics[width=0.95\linewidth]{figs/spektrum_probe.png}
        \caption{Am HPGe-Detektor gemessenes Langzeit-Spektrum mit der Bodenprobe. Der letzte Kanal am rechten Bildrand ist außen vor gelassen worden, da dieser als \enquote{Overflow-Bin} alle Ereignisse mit einer Energie außerhalb des MCA-Messbereichs zählt und das gesamte Spektrum deutlich dominieren würde.}
        \label{fig:spektrum_langzeitmessung_probe}
    \end{figure}

    \begin{figure}[H]
        \centering
        \includegraphics[width=0.95\linewidth]{figs/spektrum_hintergrund.png}
        \caption{Am HPGe-Detektor gemessenes Langzeit-Spektrum ohne die Bodenprobe. Der letzte Kanal am rechten Bildrand ist außen vor gelassen worden, da dieser als \enquote{Overflow-Bin} alle Ereignisse mit einer Energie außerhalb des MCA-Messbereichs zählt und das gesamte Spektrum deutlich dominieren würde.}
        \label{fig:spektrum_langzeitmessung_hintergrund}
    \end{figure}

    \begin{figure}[H]
        \centering
        \includegraphics[width=1\columnwidth]{figs/spektrum_differenz.png}
        \caption{Differenz der Langzeitmessungs-Spektren mit und ohne Bodenprobe ohne Anpassungsfunktionen. Der letzte Kanal am rechten Bildrand ist außen vor gelassen worden, da dieser als \enquote{Overflow-Bin} alle Ereignisse mit einer Energie außerhalb des MCA-Messbereichs zählt und das gesamte Spektrum dominieren würde.}
        \label{fig:spektrum_langzeitmessung_differenz}
    \end{figure}
    
    
    \begin{figure}[H]
        \centering
        \begin{subfigure}{0.49\textwidth}
            \includegraphics[width=\linewidth]{figs/bodenprobe_zoomed/spektrum_differenz_fit_02419.png}
        \end{subfigure}
        \hspace{0cm}
        \begin{subfigure}{0.49\textwidth}
            \includegraphics[width=\linewidth]{figs/bodenprobe_zoomed/spektrum_differenz_fit_02994.png}
        \end{subfigure}
        \caption{Vergrößerte Ausschnitte aus dem HPGe-Differenzspektrum der Bodenprobe (1).}
        \label{fig:spektrum_bodenprobe_zoom_1}
    \end{figure}

    \begin{figure}[H]
        \centering
        \begin{subfigure}{0.49\textwidth}
            \includegraphics[width=\linewidth]{figs/bodenprobe_zoomed/spektrum_differenz_fit_03431.png}
        \end{subfigure}
        \hspace{0cm}
        \begin{subfigure}{0.49\textwidth}
            \includegraphics[width=\linewidth]{figs/bodenprobe_zoomed/spektrum_differenz_fit_03568.png}
        \end{subfigure}
        \caption{Vergrößerte Ausschnitte aus dem HPGe-Differenzspektrum der Bodenprobe (2).}
        \label{fig:spektrum_bodenprobe_zoom_2}
    \end{figure}
    
    \begin{figure}[H]
        \centering
        \begin{subfigure}{0.49\textwidth}
            \includegraphics[width=\linewidth]{figs/bodenprobe_zoomed/spektrum_differenz_fit_05179.png}
        \end{subfigure}
        \hspace{0cm}
        \begin{subfigure}{0.49\textwidth}
            \includegraphics[width=\linewidth]{figs/bodenprobe_zoomed/spektrum_differenz_fit_05913.png}
        \end{subfigure}
        \caption{Vergrößerte Ausschnitte aus dem HPGe-Differenzspektrum der Bodenprobe (3).}
        \label{fig:spektrum_bodenprobe_zoom_3}
    \end{figure}
    
    \begin{figure}[H]
        \centering
        \begin{subfigure}{0.49\textwidth}
            \includegraphics[width=\linewidth]{figs/bodenprobe_zoomed/spektrum_differenz_fit_06179.png}
        \end{subfigure}
        \hspace{0cm}
        \begin{subfigure}{0.49\textwidth}
            \includegraphics[width=\linewidth]{figs/bodenprobe_zoomed/spektrum_differenz_fit_06710.png}
        \end{subfigure}
        \caption{Vergrößerte Ausschnitte aus dem HPGe-Differenzspektrum der Bodenprobe (4).}
        \label{fig:spektrum_bodenprobe_zoom_4}
    \end{figure}
    
    \begin{figure}[H]
        \centering
        \begin{subfigure}{0.49\textwidth}
            \includegraphics[width=\linewidth]{figs/bodenprobe_zoomed/spektrum_differenz_fit_09241.png}
        \end{subfigure}
        \hspace{0cm}
        \begin{subfigure}{0.49\textwidth}
            \includegraphics[width=\linewidth]{figs/bodenprobe_zoomed/spektrum_differenz_fit_09827.png}
        \end{subfigure}
        \caption{Vergrößerte Ausschnitte aus dem HPGe-Differenzspektrum der Bodenprobe (5).}
        \label{fig:spektrum_bodenprobe_zoom_5}
    \end{figure}
    
    \begin{figure}[H]
        \centering
        \begin{subfigure}{0.49\textwidth}
            \includegraphics[width=\linewidth]{figs/bodenprobe_zoomed/spektrum_differenz_fit_11361.png}
        \end{subfigure}
        \hspace{0cm}
        \begin{subfigure}{0.49\textwidth}
            \includegraphics[width=\linewidth]{figs/bodenprobe_zoomed/spektrum_differenz_fit_14814.png}
        \end{subfigure}
        \caption{Vergrößerte Ausschnitte aus dem HPGe-Differenzspektrum der Bodenprobe (6).}
        \label{fig:spektrum_bodenprobe_zoom_6}
    \end{figure}



    \begin{table}[H]
        \centering
        \large
        \begin{tabular}{cccccc}
            \toprule
            Peak-Index & A / 1 & $\mu$ / Kanal & $\sigma_{FWHM}$ / Kanal & a / Kanal$^{-1}$ & b / 1 \\
            \midrule
                0 & 4900(200) & 2419,3(3) & 12,5(6) & 0,02(6) & 0(140) \\
                1 & 1200(170) & 2994,4(8) & 12,9(1,9) & 0,06(8) & -100(300) \\
                2 & 1170(170) & 3431,1(1,2) & 19(3) & -0,08(4) & 300(140) \\
                3 & 2580(170) & 3568,6(4) & 14,8(1,0) & -0,02(2) & 100(90) \\
                4 & 810(130) & 5179,2(1,2) & 17(3) & 0,009(18) & -40(90) \\
                5 & 1710(130) & 5913,9(5) & 15,2(1,1) & -0,016(16) & 100(90) \\
                6 & 1930(130) & 6179,0(5) & 14,6(1,0) & 0,002(15) & 0(90) \\
                7 & 330(70) & 6710,4(1,1) & 12(3) & 0,021(14) & -130(100) \\
                8 & 990(100) & 9241,6(7) & 13,7(1,4) & 0,007(11) & -60(100) \\
                9 & 890(90) & 9827,3(9) & 18,9(2,0) & -0,007(11) & 70(110) \\
                10 & 340(70) & 11361,4(1,1) & 11(2) & 0,009(17) & -90(200) \\
                11 & 4900(200) & 14814,7(3) & 16,9(6) & -0,010(6) & 150(80) \\
            \bottomrule
        \end{tabular}
        \normalsize
        \caption{Anpassungsparameter der an das Differenzspektrum der Langzeitmessung (Abbildung \ref{fig:spektrum_langzeitmessung_differenz}) angepassten Gauß-Glocken mit linearem Hitnergrund (vgl. Gleichung \ref{eq:Gauss_mit_Bg}). Der Peak-Index nummeriert die einzelnen Maxima im Bild von Links nach rechts durch.}
        \label{tab:langzeit_fit_params}
    \end{table}