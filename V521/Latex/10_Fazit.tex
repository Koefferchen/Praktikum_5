\section{Fazit} 
    In diesem Versuch sind anhand von Messungen bekannter radioaktiver Proben und einer Langzeitmessung einer Bodenprobe die Unterschiede eines NaJ(Tl)-Szintillations- und eines HPGe-Halbleiterdetektors in der $\gamma$-Spektroskopie untersucht worden. Nach einer erfolgreichen Justage der Detektoren und Messelektronik ist mit ihnen je ein Zerfallsspektrum einer $^{60}$Co-Probe, eine $^{137}$Cs-Probe und einer $^{152}$Eu-Probe aufgenommen worden.
    
    Mit diesen Spektren sind die beiden Detektoren auf wesentliche Ähnlichkeiten und Unterschiede untersucht worden, wobei der Fokus auf der Energieauflösung, dem Peak-to-Total-Verhältnis und der energieabhängigen Effizienz lag. Die beobachtete Energieunsicherheit des HPGe-Detektors ist um einen Faktor $~30$ kleiner als beim NaJ-Detektor (vgl. Abb. \ref{fig:FWHM_Funktion_HPGe}, \ref{fig:FWHM_Funktion_NaI}), jedoch weist der NaJ-Szintillator eine um einen Faktor $~2$ höhere Detektionseffizienz als der HPGe-Detektor auf (vgl. Abb. \ref{fig:Effizienzkurve_HPGe}, \ref{fig:Effizienzkurve_NaI}). Ebenso erreicht der NaJ-Detektor ein besseres Peak-to-Total-Verhältnis für die $\gamma$-Linien von Caesium und Cobalt (vgl. Gl. \ref{eq:peak_to_total}). Für beide Detektoren konnte die erwartete sinkende Detektionseffizienz mit steigender Energie nachgewiesen werden. Im besonders niederenergetischen Bereich sollte die Effizienz noch einmal sinken, jedoch konnte die Konfiguration des Vresuchsaufbaus diesen Bereich mit den vorhandenen Quellen nicht sinnvoll ausmessen.
    
    Als letztes ist mit dem HPGe-Detektor eine Bodenprobe auf radioaktive Kontamination untersucht worden. Dafür wurde eine Langzeitmessung durchgeführt, bei der die Probe direkt vor dem Detektor lag, sowie eine Hintergrundmessung ohne Probe. Der Abzug dieses Hintergrunds erlaubt Rückschlüsse auf den Beiträge der Probe. Das gefundene Differenzspektrum ist in Abb. \ref{fig:spektrum_langzeitmessung_differenz_fit} mit 12 an die klarsten Maxima angepassten Kurven gezeigt. Die gefundenen $\gamma$-Energien sind nun vorsichtig mit entsprechenden Literaturwerten aus \cite{nudat3} verglichen worden, um sie ihren jeweiligen Quellen zuzuordnen. Diese Analyse ergab insbesondere, dass die verwendete Bodenprobe einige übliche Isotope aus der Uran-Actinium- und Thorium-Reihe sowie radioaktives Kalium und Cäsium enthält. Damit entspricht die radioaktive Belastung der qualitativen Erwartung.