\section{Einführung}
    Der nachfolgende Versuch beschäftigt sich mit den Eigenschaften von Szintillations- und Halbleiter-Detektoren sowie deren Anwendung in der $\gamma$-Spektroskopie. Hierbei sollen Energie-Auflösung und Effizienz eines NaI(Ta)-Szintillations-Detektors und eines HPGe-Halbleiter-Detektor anhand bekannter Proben untersucht werden. Der HPGe-Detektor wird schließlich zur Spektroskopie einer Bodenprobe verwendet, aus welcher die natürlichen Zerfallsreihen nachgewiesen werden sollen. Das Ziel des Versuchs ist ein quanitativer Vergleich beider Dektortypen. 


\subsection{Szintillations-Spektrometer}
    Beim Szintillations-Spektrometer handelt es sich um einen Detektor, der aus einem Szintillator, einer \textit{Photomultiplier Tube} (kurz: PMT) und einer Auslese-Elektronik (Basis) besteht. Fällt ionisierende Strahlung auf den mit Thallium dotierten NaJ-Szintillationskristall, werden dort Elektronen ionisiert, welche bei stufenweiser Abregung Photonen im sichtbaren Bereich emittieren. Diese Photonen fallen auf die Photokathode der PMT und lösen dabei mit jeder Iteration einen größer-werdenden Schauer aus Sekundär-Elektronen aus, welcher schließlich in der Basis verstärkt und in einen Spannungs-Impuls umgewandelt wird. Die Amplitude, bzw. die Fläche unter dem Spannungs-Impuls ist proportional zur Energie der gemessenen Strahlung und ermöglicht so die Spektroskopie einer radioaktiven $\gamma$-Quelle. Das Schema eines solchen Detektionsvorgangs ist in Abbildung \ref{fig:Aufbau_Szintillator} illustriert. 
    
    \begin{figure}[H]
        \centering
        \includegraphics[width=0.7\linewidth]{figs/Aufbau_Szintillator.jpg}
        \caption{Schematischer Aufbaus eines Szintillations-Detektors \cite{Aufbau_Szintillator}. Eintreffende Strahlung wird in sichtbares Licht umgewandelt und durch die PMT in einen verstärkten Spannungsimpuls umgewandelt.}
        \label{fig:Aufbau_Szintillator}
    \end{figure}

    Der Vorteil des verwendeten Szintillations-Spektrometers mit dem Szintillator NaI(Ta) ist der hohe Wechselwirkungs-Querschnitt $\sigma \sub{photo} \propto Z^5$ zur Detektion von Strahlung aufgrund der hohen Ordnungszahl $Z$ von Iod. Die zugehörige energieabhängige Effizienz $\epsilon(E_\gamma)$ wird als Verhältnis zwischen der Detektionsrate in einem Photopeak $N\sub{detect}$ und der Rate von Photonen, welche den Detektor erreichen ($N \sub{reach}$), gebildet. Für einen Detektor mit Durchmesser $d$ und Abstand $D$ zur Quelle der Aktivität $B$ ergibt sich über Betrachtung des Raumwinkels die absolute Effizienz \cite{Praktikumsanleitung}
    \begin{align} \label{eq:abs_effizienz}
        \epsilon (E_\gamma) = 
        \frac{N\sub{detect}}{N\sub{reach}} \approx
        \frac{N\sub{detect}}{B} \cdot \frac{4 \pi D^2}{\pi d^2} 
        \qquad \text{.}
    \end{align}
    Hierbei wird die Fläche der Detektions als Kreisscheibe der Fläche $\pi d^2$ angenommen und die Quelle als isotroper Strahler ohne Rückstreuung in der Umgebung genähert. Besitzt eine Quelle mehrere Photopeaks, so muss die Aktivität $B$ für den jeweiligen Zerfallskanal nach unten korrigiert werden. Die Anzahl der in einem Photopeak gemessenen Ereignissen entspricht der Fläche der dazu angepassten Gaußkurve $A$.\\
    
    Der Szintillaitons-Detektor benötigt keinerlei Kühlung und kann problemlos bei Raumtemperatur betrieben werden. Ein großer Nachteil des Szintillations-Detektors ist jedoch seine geringe Energie-Auflösung: Sowohl bei der Anregung der Elektronen im Szintillator als auch beim Photoeffekt an der Photo-Kathode des PMTs geht ein erheblicher Teil des Signals verloren, sodass hohe statistische Schwankungen zu einer starken Verbreiterung von Photopeaks führen. 
    
    
\subsection{Halbleiter-Detektoren}
    Halbleiter-Dektoren basieren im Gegensatz zum Szintillations-Detektor auf dem Prinzip einer PIN-Photodiode (vgl. Abb. \ref{fig:Aufbau_HalbleiterDetektor}). Dazu wird ein Block aus reinem Halbleiter zwischen zwei Schichten aus stark p- bzw. n-dotiertem Material eingefügt, sodass eine Diode mit großer Raumladungszone ensteht. Durch Anlegen einer Hochspannung in Sperr-Richtung wird diese Raumladungszone weiter vergrößert und es fließt (im Idealfall) kein Strom. Passiert nun ionisierende Strahlung den Halbleiter, so löst sie je $~\SI{3}{eV}$ Energie ein Paar aus Elektron und Elektron-Loch aus dem Material heraus, welches durch die Sperr-Spannung zu den Elektroden hin beschleunigt und somit als Strom gemessen wird. Das Strom-Signal wird durch einen Messverstärker in einen Spannungsimpuls umgewandelt, dessen Amplitude weiterhin proportional zur Energie der einfallenden Strahlung ist und die Spektroskopie ermöglicht.
    
    \begin{figure}[H]
        \centering
        \includegraphics[width=0.5\linewidth]{figs/Aufbau_HalbleiterDetektor.png}
        \caption{Schematischer Aufbaus eines Halbleiter-Detektors auf Basis von HPGe \cite{Aufbau_HalbleiterDetektor}. Eintreffende Strahlung verursacht Elektron-Loch-Paare, welche als Strom gemessen werden.}
        \label{fig:Aufbau_HalbleiterDetektor}
    \end{figure}
    
    Der große Vorteil des Halbleiter-Detektors ist seine hohe Energie-Auflösung: Aufgrund der niedrigen Schwellen-Energie ($\SI{3}{eV}$) der Wechselwirkung sind die statistischen Schwankungen in der Detektion gering und die Messung der Energie wird äußerst präzise. Der verwedendete \textit{High-Purity Germanium Detector} (HPGe-Detektor) nutzt als Halbleiter hochreines Germanium, welches aufgrund seiner hohen Reinheit in der Dioden-Anordnung geringe Leckströme durchlässt und damit ein hohes \textit{Signal-to-Noise Ratio} (SNR) zulässt. Der Detektionsmechanismus basiert erneut auf dem Photoeffekt; die geringere Ordnungszahl $Z=32$ von Germanium (Ge) gegenüber dem Iod ($Z=53$) im NaJ-Szintillator sorgt für eine verringerte Detektions-Effizienz. Ein weiterer Nachteil des Detektors ist, dass er bei hohen Spannungen mit flüssigem Stickstoff gekühlt werden muss, damit der Halbleiter intakt bleibt und Leckströme minimiert werden.
    
    Die beobachtete Halbwertsbreite eines Photopeaks $\Delta E$ am Halbleiter-Detektor setzt sich aus einem energie-unabhängigen Anteil $\Delta E_e$ sowie einem energie-abhängigen Anteil $\Delta E_d \propto \sqrt{E_\gamma}$ zusammen, welcher aus statistischen Effekten in der Ladungssammlung folgt. Es gilt \cite{Praktikumsanleitung}:
    \begin{align} \label{eq:Halbwertsbreite_HPGe}
        \Delta E = \sqrt{ \Delta E_d^2 + \Delta E_e^2 }
        = \sqrt{ x^2 \cdot E_\gamma^2 + \Delta E_e^2}
        \qquad \text{,}
    \end{align}
    wobei $x$ ein freier Parameter ist. Diese Beziehung soll im Versuch durch Betrachtung verschiedener Photopeaks nachwiesen werden. Eine weitere Metrik, anhand welcher die Detektoren verglichen werden können, ist das Peak-to-Total-Verhältnis $P\sub{T}$. Durch Division der gemessenen Ereignisse in einem Photopeak durch die insgesamt gemessenen Ereignisse spiegelt $P\sub{T}$ die Ansprechfunktion des Detektors wieder.
    

\subsection{Radioaktive Quellen}
    Im Versuch werden neben einer unbekannten Bodenprobe die Quellen $^{60}$Co, $^{137}$Cs und $^{152}$Eu verwendet und erfüllen dabei verschiedene Funktionen in der Kalibrierung des Energiebereichs $E \in [\SI{100}{keV}, \SI{1500}{keV}]$. Die Caesium-Quelle wird  durch einen einzigen Photopeak bei \SI{661.7}{keV} \cite{Cs_Energien} charakterisiert, sodass $^{137}$Cs sich in guter Näherung als monochromatischer Röntgenstrahler verwenden lässt. Die Cobalt-Quelle hingegen deckt mit ihren dominanten Emissionslinien bei den Energien \SI{1173.2}{keV} und \SI{1332.5}{keV} den Bereich hoher Energien ab \cite{Co_Energien}. Zuletzt steht eine Europiumquelle zur Verfügung, welche bei der feinen Kalibrierung von Detektoren im Röntgenbereich beliebt ist, da $^{152}$Eu durch seine vielfältigen Emissionlinien den gesamten Energiebereich von $\SI{100}{keV}$ bishin zu \SI{1400}{ke} abdeckt. Eine detaillierte Auflistung der dominantesten Emissionslinien des Europium-Isotops und deren prozentualer Anteil pro Zerfall von $^{152}$Eu ist in Tabelle \ref{tab:Eu_Literatur_Linien} abgebildet.
    \begin{table}[H]
        \centering
        \begin{tabular}{SS} \toprule
         	{$E_{lit}$ / keV} & {rel. Intens. $I$ / $\%$} \\ \midrule
        	121.7817 +- 0.0003 & 28.5300 +- 0.1600 \\
        	244.6974 +- 0.0008 & 7.5500 +- 0.0400 \\
        	344.2785 +- 0.0012 & 26.5900 +- 0.2000 \\
        	411.1165 +- 0.0012 & 2.2370 +- 0.0130 \\
        	443.9606 +- 0.0016 & 2.8270 +- 0.0140 \\
        	778.9045 +- 0.0024 & 12.9300 +- 0.0800 \\
        	867.3800 +- 0.0030 & 4.2300 +- 0.0300 \\
        	964.0570 +- 0.0050 & 14.5100 +- 0.0700 \\
        	1085.8370 +- 0.0100 & 10.1100 +- 0.0500 \\
        	1112.0760 +- 0.0030 & 13.6700 +- 0.0800 \\
        	1408.0130 +- 0.0030 & 20.8700 +- 0.0900 \\ \bottomrule
        \end{tabular}
        \caption{Energien und Intensitäten der häufigsten Gamma-Linien von $^{152}$Eu. Die Angabe der Intensitäten erfolgt in \% der Zerfälle von Europium. \cite{Praktikumsanleitung}}
        \label{tab:Eu_Literatur_Linien}
    \end{table}
    
    Die drei natürlichen Zerfallsreihen, von denen zu erwarten ist, dass sie einen Anteil am Spektrum der Bodenprobe haben, sind die Uran-Radium-Reihe ($4n+2$), die Uran-Actinium-Reihe ($4n+3$) und die Thorium-Reihe ($4n$). Die Notation $4n+x$ bezieht sich auf die Massenzahlen der in jeder Reihe vorkommenden Nuklide, da sich diese durch den $\alpha$-Zerfallsprozess nur in Viererschritten ändern kann. Neben den Quellen entlang den drei natürlichen Zerfallsketten gibt es auch instabile leichte Kerne, die häufig in der Umwelt vorkommen, ob natürlich oder menschengemacht --- etwa $^{14}$C (Notiz: kein $\gamma$-Strahler), $^{22}$Na, $^{40}$K oder $^{137}$Cs. \cite{world_nuclear_association}
    
    
    
\subsection{Vorwort}
    
    Messungen von radioaktiven Zerfällen unterliegen der Poisson-Statistik, welche vorhersagt, dass die $N$-fache Messung eines Ereignisses eine statistische Unsicherheit von $\sqrt{N}$ nach sich zieht. Zur Erhöhung der Lesbarkeit wird diese Unsicherheit zwar mit ein-berechnet, aber als Einzige nicht in den Diagrammen visualisiert. Alle Berechnungen von Unsicherheit basieren auf der Gaußschen Fehlerfortplanzung, sofern nicht anders spezifiert.\\
    
    Alle in der Versuchsdurchführung erhobenen Rohdaten und das während der Versuchsdurchführung laufend verfasste Protokoll sind auf Sciebo \cite{raw_data} erhältlich. (Link gültig bis 31.03.2026)